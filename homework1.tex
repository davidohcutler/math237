\documentclass[12pt]{article}

\usepackage[margin=1in]{geometry} 
\usepackage{amsmath,amsthm,amssymb,enumitem,bbm,xparse}

\newcommand{\N}{\mathbb{N}}
\newcommand{\Z}{\mathbb{Z}}
\newcommand{\R}{\mathbb{R}}
\newcommand{\Rd}{\mathbb{R}^{d}}
\newcommand{\exr}{[-\infty, \infty]}
\newcommand{\Biggnorm}{\Bigg | \Bigg |}

\NewDocumentCommand\closure{sm}
  {\IfBooleanTF{#1}{\overline{#2}}{\bar{#2}}}

\newenvironment{ex}[2][Exercise]{\begin{trivlist}
\item[\hskip \labelsep {\bfseries #1}\hskip \labelsep {\bfseries #2.}]}{\end{trivlist}}

\newenvironment{sol}[1][Solution]{\begin{trivlist}
\item[\hskip \labelsep {\bfseries #1:}]}{\end{trivlist}}

\newenvironment{theorem}[2][Theorem]{\begin{trivlist}
\item[\hskip \labelsep {\bfseries #1}\hskip \labelsep {\bfseries #2.}]}{\end{trivlist}}
    
\newenvironment{lemma}[2][Lemma]{\begin{trivlist}
\item[\hskip \labelsep {\bfseries #1}\hskip \labelsep {\bfseries #2.}]}{\end{trivlist}}
    
\newenvironment{definition}[2][Definition]{\begin{trivlist}
\item[\hskip \labelsep {\bfseries #1}\hskip \labelsep {\bfseries #2.}]}{\end{trivlist}}
    
\newenvironment{example}[1][Example]{\begin{trivlist}
\item[\hskip \labelsep {\bfseries #1:}]}{\end{trivlist}}
\begin{document}
\noindent David Owen Horace Cutler \hfill {\Large Math 237: Homework 1} \hfill \today

\begin{ex}{2}
    Let $E$ be a normed space over the reals. A hyper-plane $H$ is a subspace of $E$ such that the quotient space $E/H$ has dimension 1.
    \begin{enumerate}[label=(\alph*)]
        \item Show that the closure of any subspace of $E$ is again a subspace of $E$ and conclude that a hyper-plane $H$ is either closed or dense in $E$.
        \begin{proof}
            Consider some vector subspace of $E$, say $H$. We recall the definition of the closure of $H$, denoted $\closure{H}$, as the following:
            \begin{equation}
                \closure{H} = \{H \cup \text{ the limit points of } H\}
            \end{equation}
            To show that $\closure{H}$ is a vector subspace of $E$ then, we need to verify that it is closed under vector addition and scalar multiplication, and moreover that it contains the zero vector of $H$. \\ \\
            For this, note that the zero vector $0 \in \closure{H}$, as $0 \in H \subseteq \closure{H}$. Thus for $c \in \mathbb{R}$ and $a, b \in \closure{H}$, it is sufficient to show that:
            \begin{equation}
                ca + b \in \closure{H}
            \end{equation}
            For this, simply note that every point in $\closure{H}$ can be expressed as limit of points in $H$, (either a constant sequence or some non-trivial sequence for those points in $\closure{H} \setminus H$). \\ \\
            With this in mind, we represent $a$ as $\underset{n \rightarrow \infty}{\lim} a_n$ and $b$ as $\underset{n \rightarrow \infty}{\lim} b_n$ respectively, where $\{a_n\}_{n = 1}^\infty \subseteq H$ and $\{b_n\}_{n = 1}^\infty \subseteq H$. Using limit rules, we get the following equivalent convergent sequences:
            \begin{equation}
                \begin{aligned}
                    ca + b = c\underset{n \rightarrow \infty}{\lim} a_n + \underset{n \rightarrow \infty}{\lim} b_n = \underset{n \rightarrow \infty}{\lim} ca_n + b_n
                \end{aligned}
            \end{equation}
            But then of course the last sequence in (3) is a sequence of points in $H$ that converges to $ca + b$, so it must be $ca + b \in \closure{H}$. Thus $\closure{H}$ is a vector subspace of $E$. \\ \\
            We want to conclude then that a hyper-plane $H$ is either closed or dense in $E$. For this, assume $H$ is not closed. We would like to show $H$ is dense in $E$, which is tantamount to showing $E \subseteq \closure{H}$. \\ \\
            For this, consider $y \in E$. As $H$ is not closed, we have: $$\exists \{x_n\}_{n = 1}^\infty \subseteq H \text{ such that } x_n \rightarrow \hat{x} \notin H, \text{ i.e. } \hat{x} \in \closure{H} \setminus H$$ Then as $\hat{x} \notin H$, the coset $\hat{x} + H$ is not the zero vector in the quotient space. As $E/H$ is 1-dimensional then, it is spanned by $\hat{x} + H$. \\ \\
            In particular, we have for some $c \in \mathbb{R}$ that: $$y + H = c(\hat{x} + H) = c\hat{x} + H$$ Thus $y - c\hat{x} = h \in H \subseteq \closure{H}$, so we can reorganize to get $y = c\hat{x} + h$. But as $\closure{H}$ is a vector subspace, $$c\hat{x} + h = y \in \closure{H} \text{ given that } c\hat{x}, h \in \closure{H}$$ Thus $E \subseteq \closure{H}$, where we already have $\closure{H} \subseteq E$. Therefore $E = \closure{H}$, which is exactly that $H$ is dense in $E$. \\ \\
            Therefore $H$ is either closed or dense in $E$.
        \end{proof}
        \item Let $u$ be a linear functional on $E$. Prove that $u$ is discontinuous if and only if there exists a sequence $\{x_n\}_{n = 1}^\infty \subseteq E$ that conveges to $0$ and for which $u(x_n) = 1$, $\forall n \in \mathbb{N}$.
        \begin{proof}
            We will show the forward implication first. For this, we assume $u$ is discontinuous. As $u$ is a linear functional then, it must be discontinuous at $0$, i.e. we have a sequence $\{y_n\}_{n = 1}^\infty \subseteq E$ such that $y_n \rightarrow 0$ but $u(y_n)$ does not converge to $0$.
            \\ \\
            As $u(y_n)$ does not converge to zero, we have for some $\epsilon > 0$ that there is a subsequence $\{y_{n_k}\}_{k = 1}^\infty \subseteq E$ such that $|u(y_{n_k})| \geq \epsilon$ for all $k \in \mathbb{N}$. Define then a sequence $\{x_k\}_{k = 1}^\infty \subseteq E$ by stipulating $x_k = \frac{y_{n_k}}{u(y_{n_k})}$. We quickly verify $x \rightarrow 0$:
            \begin{equation}
                \begin{aligned}
                    ||x_k|| = \Biggnorm \frac{y_{n_k}}{u(y_{n_k})} \Biggnorm = \frac{1}{|u(y_{n_k})|}||y_{n_k}|| \leq \frac{1}{\epsilon}||y_{n_k}||
                \end{aligned}
            \end{equation}
            Of course then, we can take $||y_{n_k}||$ small (as the subsequence still converges to $0$), which shows $x_k \rightarrow 0$. We want to verify now that $u(x_k) = 1$ for all $k \in \mathbb{N}$. Here, we note:
            \begin{equation}
                u(x_k) = u \Bigg ( \frac{y_{n_k}}{u(y_{n_k})} \Bigg) = \frac{1}{u(y_{n_k})} u(y_{n_k}) = 1
            \end{equation}
            So $x_k$ is the desired sequence, and so the forward implication holds. \\ \\
            We note then that the reverse implication is obvious, as the negation of continuity at $0$ is that there is some sequence $\{x_n\}_{n = 1}^\infty \subseteq E$ such that $x_n \rightarrow 0$ but $u(x_n)$ does not converge to $u(0) = 0$. \\ \\
             If we assume then that we have some sequence such that $x_n \rightarrow 0$ but $u(x_n) = 1, \; \forall n \in \mathbb{N}$, it is clear that such a sequence fulfills this definition of the negation of continuity, and so $u$ would be discontinuous.
        \end{proof}
        \item Let $x_0 \in E$ be a unit norm vector, and $H$ the complement of the one dimensional space spanned by $x_0$. Show that every $x \in E$ can be uniquely decomposed as:
        \begin{equation}
            x = t(x)x_0 + y(x)
        \end{equation}
        where $t$ and $y$ are linear maps from $E$ to $\mathbb{R}$ and $H$ respectively. Prove that $t$ and $y$ are continuous iff $H$ is closed.
        \begin{proof}
            We adopt the notation $\langle x_0 \rangle = \text{span}(x_0)$. As $H$ is selected to be the complement of $\langle x_0 \rangle$, we have that: $$E = \langle x_0 \rangle \oplus H$$ From this it follows we can uniquely write any element $x \in E$ as: $$x = \lambda x_0 + h$$ for some $\lambda \in \mathbb{R}, h \in H$. We denote these elements $\lambda_x$ and $h_x$, i.e. $x = \lambda_x x_0 + h_x$. With this in mind, we define our functions $t(x) = \lambda_x$ and $y(x) = h_x$.
            \begin{enumerate}[label=(\arabic*)]
            \item \textit{(Decomposition and well-definedness)}
            By construction, we then get a decomposition $x = t(x)x_0 + y(x)$ for all $x$. \\ \\Moreover, note these functions are well-defined, as the decomposition of $x$ into a sum of elements in complementary subspaces is unique.
            \item \textit{(Linearity)}
            Linearity also follows almost immediately from the uniqueness of our decomposition. In particular, we note for $x, y \in E$, we of course can write $$x = \lambda_x x_0 + h_x \text{ and } y = \lambda_y x_0 + h_y$$ Moreover, we can express $x + y = \lambda_{x+y}x_0 + h_{x + y}$, for some $\lambda_{x + y} \in \mathbb{R}$ and $h_{x_y} \in H$. \\ \\
            However, we also have that: $$x + y = \lambda_x x_0 + h_x + \lambda_y x_0 + h_y = (\lambda_x + \lambda_y)x_0 + (h_x + h_y)$$ Uniqueness of this decomposition gives us then that: $$\lambda_{x + y} = \lambda_{x} + \lambda_y \text{ and } h_{x + y} = h_x + h_y$$ Thus $t(x + y) = t(x) + t(y)$ and $y(x + y) = y(x) + y(y)$, and so both functions are linear. 
            \item \textit{(Uniqueness)}
            We finally remark that these functions $t, y$ are necesarily unique, again using the uniqueness of the decomposition into a sum. \\ \\
            In particular, we know we have the unique decomposition $x = \lambda_x x_0 + h_x$, so if we have functions: $$a : E \rightarrow \mathbb{R} \text{ and } b : E \rightarrow H \text{ where } x = a(x)x_0 + b(x)$$ as $a(x)x_0 \in \langle x_0 \rangle$ and $b(x) \in H$, the only possibility is that $a(x) = \lambda_x$ and $b(x) = h_x$, which yields the given functions $t$ and $y$.
            \item \textit{(Continuity of both implies closedness of H)}
            We want now to show that $t$ and $y$ being continuous implies the closedness of $H$. \\ \\ 
            For this, we will consider the contrapositive, that is $H$ being not closed implies that $t$ or $y$ is not continuous. In particular, we will show that $t$ is not continuous if $H$ is not closed. \\ \\
            For this, assume that we have a sequence: $$\{x_n\}_{n = 1}^\infty \subseteq H \text{ such that } x_n \rightarrow x \notin H$$ We note as $x \notin H$, it must be then that $\lambda_x \neq 0$, i.e. $t(x) \neq 0$. However, as each $x_n \in H$, we know $t(x_n) = \lambda_{x_n} = 0$. Thus: 
            $$\underset{n \rightarrow \infty}{\lim} t(x_n) = 0 \neq 1 = t(x)$$
            So $t$ is not continuous at $x$, the desired result.
            \item \textit{(Closedness of H implies continuity of both t and y)}
            Note it is enough to prove that one is continuous, i.e. $t$ is continuous, because we can reorganize to get $y(x) = x - t(x)x_0$. \\ \\ In particular, $t : E \rightarrow \mathbb{R}$ being continuous has $tx_0 : E \rightarrow E$ being continuous, which has $y$ continuous as the difference of the identity and this function. \\ \\
            So we want to show $H$ being closed has $t$ continuous. For this, we assume for the sake of contradiction that $t$ is discontinuous. \\ \\Then as $t$ is a linear functional on $E$, we apply our result in (b) to get some sequence: $$\{x_n\}_{n = 1}^\infty \subseteq E \text{ such that } x_n \rightarrow 0 \text{ but }t(x_n) = 1$$ for all $n \in \mathbb{N}$. \\ \\
            Then we get $y(x_n) = x_n - x_0$. So taking the limit gets $\underset{n \rightarrow \infty}{\lim} y(x_n) = -x_0$. But this is an issue, as $y$ maps into $H$, so $\{y(x_n)\}_{n = 1}^\infty \subseteq H$. \\ \\  As $H$ is closed then, it should be $-x_0 \in H$, but of course $-x_0 \neq 0 \in \langle x_0 \rangle$ and so as $H$ is a complementary subspace, it must be then that $-x_0 \notin H$. \\ \\ So we have a contradiction, and thus $H$ being closed has $t$ and furthermore $y$ continuous.
            \end{enumerate}
        \end{proof}
        \item Let $u$ be a linear functional on $E$. Prove that $u$ is continuous if and only if $H$, the kernel, is closed.
        \begin{proof}
            We will show both directions seperately.
            \begin{enumerate}[label=(\arabic*)]
                \item \textit{(Continuity of u implies the kernel is closed)} Say $u$ is continuous, and assume for contradiction that $H = \text{ker}(u)$ is not closed. \\ \\
                By definition then, there exists some sequence $$\{x_n\}_{n = 1}^\infty \subseteq \text{ker}(u) \text{ such that } x_n \rightarrow x \notin \text{ker}(u)\text{, i.e. } u(x) \neq 0$$ We note additionally then that for any $n \in \mathbb{N}$, we have that $u(x_n) = 0$, as each term in in the kernel.
                \\ \\
                We can apply continuity then:
                \begin{equation}
                    0 = \underset{n \rightarrow \infty}{\lim} u(x_n) = u(\underset{n \rightarrow \infty}{\lim} x_n) = u(x) \neq 0
                \end{equation}
                But of course this generates a contradiction. So it must be $\text{ker}(u)$ is closed.
                \item \textit{(The closedness of the kernel implies the continuity of u)} Say that $\text{ker}(u)$ is closed, and assume for contradiction that $u$ is not continuous. \\ \\
                As $u$ is not continuous, it is not bounded, i.e. for $\forall c > 0$, there exists $x \in E$ such that:
                $$|u(x)| > c||x||$$
                Consider the sequence $\{x_n\}_{n = 1}^\infty$ generated by taking $c = n$ in the above definition, i.e. $|u(x_n)| \geq n||x_n||$. \\ \\Take then some point $x \notin \text{ker}(u)$ (if no such point exists, then $u$ is identically zero and thus already continuous). We define the following sequence:
                \begin{equation}
                    \hat{x}_n = x - \frac{u(x)}{u(x_n)}x_n
                \end{equation}
                Note then: 
                \begin{equation}
                    \begin{aligned}
                    \hat{x}_n = u \Bigg (x - \frac{u(x)}{u(x_n)}x_n \Bigg ) = u(x) - u \Bigg ( \frac{u(x)}{u(x_n)}x_n \Bigg ) \\
                    = u(x) - \frac{u(x)}{u(x_n)}u(x_n) = u(x) - u(x) = 0
                    \end{aligned}
                \end{equation}
                Thus $\{\hat{x}_n\}_{n = 1}^\infty \subseteq \text{ker}(u)$. We claim then that $\frac{u(x)}{u(x_n)}x_n \rightarrow 0$ as $n \rightarrow \infty$. For this we note the following argument:
                \begin{equation}
                    \begin{aligned}
                        \Biggnorm \frac{u(x)}{u(x_n)}x_n \Biggnorm = \Bigg | \frac{u(x)}{u(x_n)} \Bigg | ||x_n|| = \frac{|u(x)|}{|u(x_n)|} ||x_n|| \\
                        = \frac{||x_n||}{|u(x_n)|}|u(x)| < \frac{|u(x)|}{n}
                    \end{aligned}
                \end{equation}
                Taking $n$ large thus demonstrates that $\frac{u(x)}{u(x_n)}x_n \rightarrow 0$ as $n \rightarrow \infty$. \\ \\ Thus $\underset{n \rightarrow \infty}{\lim} \hat{x}_n = x \notin \text{ker}(u)$, but this is a contradiction as $\{\hat{x}_n\}_{n = 1}^\infty \subseteq \text{ker}(u)$ and $\text{ker}(u)$ is closed. So it must be $u$ is continuous.
            \end{enumerate}
        \end{proof}
    \end{enumerate}
\end{ex}

\begin{ex}{5}
    Let $E$ be a Banach space.
    \begin{enumerate}[label=(\alph*)]
        \item Prove that if $T \in L(E,E)$ and $||I - T|| < 1$ where $I$ is the identity operator, then $T$ is invertible, and that the series $\sum_{n = 0}^\infty (I - T)^n$ converges in $L(E,E)$ to $T^{-1}$.
        \begin{proof}
            For simplicitly, we denote $S = I - T$, and observe clearly $S \in L(E,E)$. As $E$ is complete with respect to its norm metric, so is $L(E,E)$ by a theorem in Folland. \\ \\
            Note then, using submultiplicity of the norm metric, we get the following:
            \begin{equation}
                \sum_{n = 0}^\infty ||S^n|| \leq \sum_{n = 0}^\infty ||S||^n < \infty
            \end{equation}
            In particular, convergence here follows from the geometric series test, as $||S|| < 1$.\\ \\ As $L(E,E)$ is complete then, we have that the series $\sum_{n = 0}^\infty S^n$ converges to a $T^{-1} \in L(E,E)$ (we are not assuming this is the inverse of $T$, just calling it this for now). \\ \\
            We want then to actually verify this $T^{-1} = \sum_{n = 0}^\infty S^n \in L(E,E)$ is the inverse of $T$. For this, we need to show composition on both sides yields the identity operator $I$. \\ \\
            We start by verifying it is a right inverse, using the linearity of $T$:
            \begin{equation}
                \begin{aligned}
                T\sum_{n = 0}^\infty S^n = T\underset{k \rightarrow \infty}{\lim} \sum_{n = 0}^k S^n = \underset{k \rightarrow \infty}{\lim} \sum_{n = 0}^k TS^n = \underset{k \rightarrow \infty}{\lim} \sum_{n = 0}^k (I - S)S^n \\
                = \underset{k \rightarrow \infty}{\lim} \sum_{n = 0}^k S^n - S^{n + 1} = \underset{k \rightarrow \infty}{\lim} I - S^{k + 1}
                \end{aligned}
            \end{equation}
            Using the divergence test on the first series in (11), we note that it is that $\underset{k \rightarrow \infty}{\lim} ||S^k|| = 0$, which is that $S^k \xrightarrow{L(E,E)} 0$, where $0$ is the zero operator. \\ \\Thus the last limit in (12) is just the identity, and thus $\sum_{n = 0}^\infty S^n$ is a right inverse. \\ \\
            For verifying that it also left inverse, we note the same argument works as $(I - S)S^n = S^n(I - S)$, and so the argument ends being the exact same. \\ \\ Thus $T$ is invertible with $T^{-1} = \sum_{n = 0}^\infty S^n \in L(E,E)$, and of course by construction our series $\sum_{n = 0}^\infty S^n \in L(E,E)$ converges to $T^{-1}$.
            \end{proof}
        \item Show that if $T \in L(E,E)$ is invertible and $||S - T|| < ||T^{-1}||^{-1}$, then $S$ is invertible. Conclude that the set of invertible operators in $L(E,E)$ is open.
        \begin{proof}
            We get the following use submultiplicity of the operator norm:
            \begin{equation}
                ||T^{-1}S - I|| = ||T^{-1}(S - T)|| \leq ||T^{-1}||\:||S - T|| < ||T^{-1}||\:||T^{-1}||^{-1} = 1
            \end{equation}
            Using part (a) then, it follows that $T^{-1}S$ is invertible, i.e. there is some $R \in L(E,E)$ such that $RT^{-1}S = T^{-1}SR = I$.  \\ \\
            Of course, this shows $RT^{-1}$ is a left inverse of $S$. Given $T^{-1}SR = I$ we can reorganize:
            \begin{equation}
                T^{-1}SR = I \rightarrow SR = T \rightarrow SRT^{-1} = I
            \end{equation}
            So $S$ is invertible with inverse $RT^{-1}$. \\ \\
            For the conclusion that the set of invertible operators is open in $L(E,E)$, we just note that this has for an invertible operator $T$ that $S \in B_{||T^{-1}||^{-1}}(T)$ is also invertible, i.e.: $$B_{||T^{-1}||^{-1}}(T) \subseteq \{\text{invertible operators in } L(E,E)\}$$ but this is precisely the definition of openness is the norm topology.
        \end{proof}
    \end{enumerate}
\end{ex}

\begin{ex}{8}
    Suppose that $H$ is a Hilbert space and $T \in L(H,H)$.
    \begin{enumerate}[label=(\alph*)]
        \item Prove that there exists a unique $T^* \in L(H,H)$, called the adjoint of $T$, such that $\langle Tx, y \rangle = \langle x, T^*y \rangle$, for all $x, y \in H$.
        \begin{proof}
            Let $y \in H$, then $\langle Tx, y \rangle$ is a functional in $x$. Linearity of this functional immediately follows from the linearity of the inner product and $T$. \\ \\
            We can furthermore see it is bounded by the Cauchy-Schwarz inequality (and that $T$ is bounded, i.e. $||Tx|| \leq c||x||$ for $c > 0$):
            \begin{equation}
                |\langle Tx, y \rangle| \leq ||Tx||\:||y|| \leq (c||y||)||x||
            \end{equation}
            And so it is bounded. As it a linear, bounded functional then, we can apply the \textbf{Riesz Representation Theorem} to get a unique $z \in H$ for which we have the following:
            \begin{equation}
                \langle Tx, y \rangle = \langle x, z \rangle, \forall x \in H
            \end{equation}
            With this in mind, we define $T^*y = z$. It is clear this mapping is unique, as $z$ for which (16) is satisfied is unique. Moreover, by construction, it is that $\langle Tx, y \rangle = \langle x, T^*y \rangle$ for all $x, y \in H$.
            \\ \\
            We need to verify this definition yields an operator that is linear and bounded. We start with linearity, we would like to show $T^*(ag + bh) = aT^*g + bT^*h$ where $g, h \in H$ and $a, b \in \mathbb{C}$. \\ \\
            For this, we note the following argument:
            \begin{equation}
                \begin{aligned}
                    \langle x, T^*(ag + bh) \rangle \\
                    = \langle Tx, ag + bh \rangle \\
                    = \closure{a}\langle Tx, g \rangle + \closure{b} \langle Tx, h \rangle \\
                    = \closure{a} \langle x, T^*g \rangle + \closure{b} \langle x, T^*h\rangle \\
                    = \langle x, aT^*g \rangle + \langle x, bT^*h\rangle \\
                    = \langle x, aT^*g + bT^*h \rangle
                \end{aligned}
            \end{equation}
            Thus:
            \begin{equation}
                \langle x, T^*(ag + bh) \rangle = \langle Tx, ag + bh \rangle = \langle x, aT^*g + bT^*h \rangle, \forall x \in H
            \end{equation}
            But as we've already shown $\langle Tx, ag + bh\rangle$ is a linear bounded functional, and so it is equivalent to $\langle x, z \rangle$ for all $x$ for some unique $z$. \\ \\
            Uniqueness of this $z$ thus tells us from (18) it is that $T^*(ag + bh) = aT^*g + bT^*h$, which is precisely linearity. \\ \\
            We need finally to show boundedness then. Here, will we again appeal to the \textbf{Riesz Representation Theorem}. Recall that by this theorem and the definition of $T^*y$, we have that $||T^*y|| = ||\langle Tx, y \rangle||$, where the latter term is the operator norm. \\ \\
            In particular, we get the following:
            \begin{equation}
                \begin{aligned}
                    ||T^*y|| = ||\langle Tx, y \rangle|| = \underset{||x|| = 1}{\sup} |\langle Tx, y\rangle| \\
                    \leq \underset{||x|| = 1}{\sup} ||Tx||\:||y|| = ||y||\:||T||
                \end{aligned}
            \end{equation} 
            So $T^*$ is bounded with $c = ||T||$, and so we finally have $T^* \in L(H, H)$. \\ \\
            Putting this all together has that we have a unique $T^*$ such that $\langle Tx, y \rangle = \langle x, T^* y \rangle$, for all $x, y \in H$, and moreover $T^* \in L(H,H)$.
        \end{proof}
        \item Prove that $T^* = V^{-1}T^\dagger V$, where $V$ is the conjugate-linear isomorphism from $H$ to $H^*$ given by $(Vy)(x) = \langle x, y \rangle$.
        \begin{proof}
            For $T \in L(H,H)$, we have $\langle x, T^*y \rangle = \langle Tx , y \rangle$, for all $x, y \in H$. Using this, we can get the following:
            \begin{equation}
                \begin{aligned}
                    \langle x, T^*y \rangle = \langle Tx , y \rangle \\
                    \rightarrow (VT^*y)(x) = \langle Tx, y \rangle
                \end{aligned}
            \end{equation}
            Moreover, we have the following:
            \begin{equation}
                \begin{aligned}
                    \langle Tx, y \rangle = (Vy)(Tx) = (Vy \circ T)(x) = T^\dagger Vy(x)
                \end{aligned}
            \end{equation}
            We use this to continue (20):
            \begin{equation}
                \begin{aligned}
                (VT^*y)(x) = \langle Tx, y \rangle \\ \rightarrow 
                (VT^*y)(x) = T^\dagger Vy(x), \text{ still for all } x \in H \\
                \rightarrow VT^*y = T^\dagger Vy, \text{ still for all } y \in H \\
                \rightarrow VT^* = T^\dagger V \\
                \rightarrow T^* = V^{-1}T^\dagger V
                \end{aligned}
            \end{equation}
            Which is the desired result.
        \end{proof}
        \item Prove that:
        \begin{enumerate}[label=(\roman*)]
            \item $||T^*|| = ||T||$
            \begin{proof}
                We will establish this by showing that $V: H \rightarrow H^*$ and $V^{-1} : H^* \rightarrow H$ as given in the previous part both have operator norm $1$. \\ \\
                We first move to compute the norm of $V$:
                \begin{equation}
                    \begin{aligned}
                        ||V|| = \underset{||y|| = 1}{\sup}||Vy||= \underset{||y|| = 1}{\sup} \underset{||x|| = 1}{\sup} |\langle x, y \rangle|
                    \end{aligned}
                \end{equation}
                On one hand: 
                \begin{equation}
                    \underset{||y|| = 1}{\sup} \underset{||x|| = 1}{\sup} |\langle x, y \rangle| \leq  \underset{||y|| = 1}{\sup} \underset{||x|| = 1}{\sup} ||x||\:||y|| = 1
                \end{equation}
                But then also clearly this quantity $||V||$ must bound above $|\langle x, x \rangle|$ for some $x$ where $||x|| = 1$, i.e. $||V|| \geq |\langle x, x \rangle| = 1$. Thus $||V|| = 1$. \\ \\
                To ascertain the value of $||V^{-1}||$, we first need to explicitly write out what $V^{-1}$ is. \\ \\For this, we claim that $V^{-1} : H^* \rightarrow H$ is the mapping given by $f \mapsto z$, where $f$ is a bounded linear functional and $z$ is the unique point associated with it by the \textbf{Riesz Representation Theorem}. \\ \\
                We verify this is indeed the inverse:
                \begin{equation}
                    \begin{aligned}
                        V^{-1}Vy = V^{-1}\langle \cdot, y \rangle = y \\
                        VV^{-1}f = Vz = \langle \cdot, z \rangle = f
                    \end{aligned}
                \end{equation}
                Note here we are making extensive use the \textbf{Riesz Representation Theorem} (in particular, the uniqueness of $z$). \\ \\
                Thus the described mapping is $V^{-1}$, and we can ascertain its norm again by using the \textbf{Riesz Representation Theorem}, in particular using the fact that: $$||f|| = ||\text{the } z \text{ associated with it}||$$
                We consequently have the following:
                \begin{equation}
                    ||V^{-1}|| = \underset{||f||=1}{\sup} ||V^{-1}f|| = \underset{||f||=1}{\sup} ||z|| = \underset{||f||=1}{\sup} ||f|| = 1
                \end{equation}
                And thus $||V^{-1}|| = 1$. Using the previous part and some reorganizing then, we have that both $T^* = V^{-1}T^\dagger V$ and $VT^{*}V^{-1} = T^\dagger$. Recalling from \textbf{Problem 6} that $||T^\dagger|| = ||T||$ then, we get the following using submultiplicity:
                \begin{equation}
                    \begin{aligned}
                    ||T^*|| =  ||V^{-1}T^\dagger V|| \leq ||V^{-1}||\:||T^\dagger||\:||V|| = ||T^\dagger|| = ||T|| \\
                    ||T|| = ||T^\dagger|| = ||VT^*V^{-1}|| \leq ||V||\:||T^{*}||\:||V^{-1}|| = ||T^{*}||
                    \end{aligned}
                \end{equation}
                Thus $||T|| = ||T^*||$.
            \end{proof}
            \item $||TT^*|| = ||T||^2$
            \begin{proof}
                Obviously, from the prior part, we get $||TT^*|| \leq ||T||^2$, as this is just:
                \begin{equation}
                    ||TT^{*}|| \leq ||T||\:||T^*|| = ||T||^2
                \end{equation}
                We want to show $||T||^2 \leq ||TT^*||$. For this, we note the following argument:
                \begin{equation}
                    \begin{aligned}
                        ||T||^2 
                        = \Big ( \underset{||x|| = 1}{\sup} ||Tx|| \Big)^2 \\
                        = \underset{||x|| = 1}{\sup} ||Tx||^2 
                        = \underset{||x|| = 1}{\sup} |\langle Tx, Tx \rangle| \\
                        = \underset{||x|| = 1}{\sup} |\langle x, T^*Tx \rangle| 
                        \leq \underset{||x|| = 1}{\sup} ||x||\:||T^*Tx|| \\
                        = \underset{||x|| = 1}{\sup} ||T^*Tx|| 
                        = ||T^*T||
                    \end{aligned}
                \end{equation} 
                Thus $||TT^*|| = ||T||^2$.
            \end{proof}
            \item $(aS + bT)^* = \closure{a}S* + \closure{b}T^*$
            \begin{proof}
                For this, we note the following argument using the sesquilinear properties of the inner product:
                \begin{equation}
                    \begin{aligned}
                        \langle x, (aS + bT)^*y \rangle = \langle (aS + bT)x, y \rangle \\
                        = \langle aSx + bTx, y \rangle = a\langle Sx, y \rangle + b\langle Tx, y \rangle \\
                        = a\langle x, S^*y\rangle + b\langle x, T^*y\rangle = \langle x, \closure{a}S^*y \rangle + \langle x, \closure{b}T^*y \rangle \\
                        = \langle x, \closure{a}S^*y + \closure{b}T^*y \rangle, \forall x, y \in H
                    \end{aligned}
                \end{equation}
                The idea given after $(17)$ concludes then that it must be $(aS + bT)^* = \closure{a}S* + \closure{b}T^*$. However, we can also see this in a more general way by the following, fixing a $y \in H$:
                \begin{equation}
                    \begin{aligned}
                    \langle x, (aS + bT)^*y \rangle = \langle x, \closure{a}S^*y + \closure{b}T^*y \rangle \\
                    \rightarrow \langle x, (aS + bT)^*y \rangle - \langle x, \closure{a}S^*y + \closure{b}T^*y \rangle = 0 \\
                    \rightarrow \langle x, (aS + bT)^*y \rangle + \langle x, -\closure{a}S^*y - \closure{b}T^*y \rangle = 0 \\
                    \rightarrow \langle x, (aS + bT)^*y -\closure{a}S^*y - \closure{b}T^*y \rangle = 0, \forall x \in H
                    \end{aligned}
                \end{equation}
                The only vector orthogonal to all others is the zero vector then, so $(aS + bT)^*y -\closure{a}S^*y - \closure{b}T^*y = 0$, $\forall y \in H$, i.e.:
                \begin{equation}
                    \begin{aligned}
                        (aS + bT)^*y = \closure{a}S^*y + \closure{b}T^*y \\
                        \rightarrow (aS + bT)^*y = (\closure{a}S^* + \closure{b}T^*)y, \forall y \in H \\
                        \rightarrow (aS + bT)^* = \closure{a}S^* + \closure{b}T^*
                    \end{aligned}
                \end{equation}
                I will use this same motif in later parts by appealing to it here.
            \end{proof}
                \item $(ST)^* = T^*S^*$
                We use the same type of argument as the previous parts:
                \begin{proof}
                    \begin{equation}
                        \begin{aligned}
                            \langle x, (ST)^*y \rangle = \langle STx, y \rangle \\
                            = \langle Tx, S^*y \rangle = \langle x, T^*S^* y \rangle 
                        \end{aligned}
                    \end{equation}
                    The same trick as the previous part gets $(ST)^* = T^*S^*$.
            \end{proof}
            \item $T^{**} = T$ Again, we use a similar argument:
            \begin{proof}
                \begin{equation}
                    \begin{aligned}
                        \langle x, T^{**}y \rangle = \langle T^*x, y \rangle \\
                        = \overline{\langle y, T^*x \rangle} = \overline{\langle Ty, x \rangle} \\
                        = \langle x, Ty \rangle
                    \end{aligned}
                \end{equation}
                The same idea again concludes $T^{**} = T$.
            \end{proof}
        \end{enumerate}
        \item Let $R(T)$ and $N(T)$ denote the range and nullspace of $T$ respectively. Prove that $R(T)^\perp = N(T^*)$ and $N(T)^\perp = \overline{R(T^*)}$.
        \begin{proof}
            We will first show that $R(T)^\perp = N(T^*)$. For this, let $x \in R(T)^\perp$. This is precisely that for $\forall y \in H$ we have $\langle x, Ty \rangle = 0$. Using that $T^{**} = T$ then, this is that:
            $$0 = \langle x, Ty \rangle = \langle x, T^{**}y \rangle = \langle T^*x, y \rangle, \forall y \in H$$
            Again, the only vector orthogonal to all other vectors is the the zero vector. Thus $T^*x = 0$, and so $x \in N(T^*)$. It follows $R(T)^\perp \subseteq N(T^*)$. \\ \\
            Consider $x \in N(T^*)$, then $T^*x = 0$. Thus $\langle T^*x, y \rangle = 0, \forall y \in H$. Continuing this we get (again using $T^{**} = T$):
            $$0 = \langle T^*x, y \rangle = \langle x, Ty \rangle, \forall y \in H$$
            As every term in $R(T)$ can be expressed as $Ty$ then for some $y \in H$, it follows that $x \in R(T)^\perp$. Thus $R(T)^\perp \subseteq N(T^*)$, and so $R(T)^\perp = N(T^*)$. \\ \\
            We want then to prove that $N(T)^\perp = \overline{R(T^*)}$. Note first, that by substituting $T^*$ in our previous work, we get $R(T^*)^\perp = N(T)$ (we can do this as $T^* \in L(H,H)$ and $T^{**} = T$). \\ \\
            Thus $\Big (R(T^*)^\perp \Big)^\perp = N(T)^\perp$. Recall then $\Big (R(T^*)^\perp \Big)^\perp = \overline{\text{span} R(T^*)}$. \\ \\
            Obviously $R(T^*) \subseteq \text{span} R(T^*)$. Consider then some $x \in \text{span} R(T^*)$, then $$x = \sum_{n = 1}^N c_nT^*y_n = T^* \Big ( \sum_{n = 1}^N c_ny_n \Big)$$ using the linearity of $T^*$. But then of course $\sum_{n = 1}^\infty c_ny_n \in H$, so it is that $x$ is in the range of $T^*$, i.e. $x \in R(T*)$. Thus $R(T^*) = \text{span} R(T^*)$. \\ \\
            It follows $\overline{R(T^*)} = \Big (R(T^*)^\perp \Big)^\perp = N(T)^\perp$, which gives the desired result.
        \end{proof}
        \item Show that $T$ is unitary if and only if $T$ is invertible and $T^{-1} = T^*$. 
        \begin{proof}
            Assume $T$ is unitary, then we have it is surjective and $\langle Tx, Ty \rangle = \langle x, y \rangle$, $\forall x, y \in H$.  Thus:
            $$\langle x, y \rangle = \langle Tx, Ty \rangle = \langle x, T^*Ty \rangle$$
            Consider some $y \in H$. Using this we get the following argument:
            \begin{equation}
                \begin{aligned}
                    \langle x, T^*Ty \rangle = \langle x, y \rangle \\
                    \rightarrow \langle x, T^*Ty \rangle - \langle x, y \rangle = 0 \\
                    \rightarrow \langle x, T^*Ty \rangle + \langle x, -y \rangle = 0 \\
                    \rightarrow \langle x, T^*Ty - y \rangle = 0, \forall x \in H
                \end{aligned}
            \end{equation}
            Using the same argument as earlier, thus $T^*Ty - y = 0 \rightarrow T^*Ty = y, \: \forall y \in H$, i.e. $T^*T = I$. \\ \\ $T$ has a left inverse then, and so it is injective. This thus has it is a bijection, and so it admits a unique inverse $T^{-1}$. \\ \\
            Furthermore, we note $T^*TT^{-1} = T^{-1} \rightarrow T^* = T^{-1}$. So $T^* \in L(H,H)$ is the inverse of $T$, and so $T$ is invertible with $T^{-1} = T^*$.
            \\ \\
            Say then for the reverse implication that $T$ is invertible and $T^{-1} = T^*$. Obviously, this has that $T$ is surjective. We want to show then that the inner product is preserved, i.e.:
            \begin{equation}
                \langle Tx, Ty \rangle = \langle x, y \rangle, \; \forall x, y \in H
            \end{equation}
            For this, we just note the following argument:
            \begin{equation}
                \begin{aligned}
                    \langle Tx, Ty \rangle = \langle x, T^*Ty \rangle = \langle x, y \rangle
                \end{aligned}
            \end{equation}
            It follows $T$ is unitary.
        \end{proof}
    \end{enumerate}
\end{ex}

\begin{ex}{12}
    Let $M$ be a closed subspace of $L^2([0,1])$ that is contained in $C([0,1])$. 
    \begin{enumerate}[label=(\alph*)]
        \item Prove that there exists a $C > 0$ such that $||f||_u \leq C||f||_2$ for all $f \in M$.
        \begin{proof}
            Consider the space $(M, ||\cdot||_u)$, we would like to show it is a closed subspace of $(C[0,1], ||\cdot||_u)$. \\ \\
            We first note that $C[0,1]$ is a subspace of $L^2{[0,1]}$ (as the continuous functions are closed under sums and scaling). \\ \\
            As $M$ is also a subspace of $L^2([0,1])$ then and $C[0,1]$ contains $M$, it naturally follows $M$ is a subspace of $C[0,1]$.
            \\ \\We want then to show $M$ is closed in $C[0,1]$ under the topology generated by the supremum norm. \\ \\
            By assumption, $M$ is closed under the norm topology generated by the $L^2$ norm, i.e.: $$\text{if } \{f_n\}_{n = 1}^\infty \subseteq M \text{ and } f_n \xrightarrow{L_2} f \in L^2[(0,1)] \text{ then } f \in M$$
            Consider then some sequence of functions $\{f_n\}_{n = 1}^\infty \subseteq M$ such that: $$f_n \rightarrow f \in C[0,1] \subseteq L^2([0,1]) \text{ uniformly}$$
            We recall uniform convergence implies convergence in $L^2$, so using the closedness of $M$ with respect to the $L^2$ norm in $L^2([0,1])$ thus implies that $f \in M$. \\ \\
            So $M$ is a closed subset of $C[0,1]$ under the topology induced by the supremum norm. \\ \\
            It follows $(M, ||\cdot||_u)$ is a closed subspace of $(C[0,1], ||\cdot||_u)$. As the latter is Banach then (a well-known result), so is the former (as closed subspaces of Banach spaces are also Banach spaces). \\ \\
            Similarly, this has that $(M, ||\cdot||_2)$ is a Banach space. With this in mind, we define a mapping $I$ by the following specifications:
            \begin{equation}
                \begin{aligned}
                    I : (M, ||\cdot||_u) \rightarrow (M, ||\cdot||_2) \\
                    I : f \mapsto f
                \end{aligned}
            \end{equation} 
            Trivially, this map is linear, as it is essentially the identity map. Moreover, we remark it is continuous as uniform convergence implies convergence in the $L^2$ norm. \\ \\
            In particular, that is whenever we have a sequence of functions $\{f_n\}_{n = 1}^\infty$ such that $f_n \rightarrow f$ uniformly, we naturally have $I(f_n) = f_n \xrightarrow{L_2} f = I(f)$.
            \\ \\
            As $I$ is continuous then, it is bounded. Moreover, it is trivially a bijection. Appealing to the \textbf{Bounded Inverse Theorem} then, it follows that $I$ has a bounded inverse. \\ \\
            Of course this inverse $I^{-1}$ is also the identity mapping, but we now know it is bounded. In particular, this means for some $C > 0$:
            \begin{equation}
                ||I^{-1}f||_u = ||f||_u \leq C||f||_2
            \end{equation}
            Which is the desired result.
        \end{proof}
        \item For each $x \in [0,1]$, prove that there exists a $g_x \in M$ such that $f(x) = \langle f, g_x \rangle$ for all $f \in M$ and that $||g_x||_2 \leq C$.
        \begin{proof}
            Let some $x \in [0,1]$. Consider the functional $T_x : (M, ||\cdot||_2) \rightarrow \mathbb{C}$ defined by $T_x : f \mapsto f(x)$. \\ \\
            We quickly verify $T_x$ is linear:
            \begin{equation}
                \begin{aligned}
                    T_x(af + bg) = (af + bg)(x) \\
                    = af(x) + bg(x) = aT_x(f) + bT_x(g)
                \end{aligned}
            \end{equation}
            Moreover, it is clearly bounded as:
            \begin{equation}
                |f(x)| \leq ||f||_u \leq C||f||_2
            \end{equation}
            per the previous part. Note then that as $M$ is a closed subspace of $L^2([0,1])$, we have that $(M, ||\cdot||_2)$ is a Hilbert space under the inherited inner product structure.
            
            Appealing to the \textbf{Riesz Representation Theorem} then, we have that there is some unique $g_x \in M$ such that:
            \begin{equation}
                f(x) = T_x(f) = \langle f, g_x \rangle,\:\forall f \in M
            \end{equation}
            Moreover, we have that:
            \begin{equation}
                ||g_x||_2 = ||T_x|| = \underset{||f||_2 = 1}{\sup} |f(x)|
            \end{equation}
            But when $||f||_2 = 1$, we naturally have:
            \begin{equation}
                |f(x)| \leq ||f||_u \leq C||f||_2 = C
            \end{equation}
            So as we have here that$|f(x)| \leq C$, it follows:
            \begin{equation}
                ||g_x||_2 = ||T_x|| = \underset{||f||_2 = 1}{\sup} |f(x)| \leq C
            \end{equation}
            Thus our $g_x$ satisifes the desired properties.
        \end{proof}
        \item Show that the dimension of $M$ is at most $C^2$ by proving that if $\{f_k\}_{k = 1}^\infty$ is any orthonormal sequence in $M$ then $\sum_{k = 1}^\infty |f_k(x)|^2 \leq C^2$ for all $x \in [0,1]$. 
        \begin{proof}
            Let $x \in [0,1]$ Using our work in part (b) (i.e. generating a $g_x$), this becomes a routine application of \textbf{Bessel's inequality}, which applies as we are given an orthonormal sequence.
            \begin{equation}
                \begin{aligned}
                    \sum_{k = 1}^\infty |f_k(x)|^2 = \sum_{k = 1}^\infty |\langle f_k, g_x \rangle| \\
                    = \sum_{k = 1}^\infty |\langle g_x, f_k \rangle| \leq ||g_x||_2^2 \leq C^2
                \end{aligned}
            \end{equation}
            Note we are able to consider this sequence as countable, as the $M$ inherits the seperability of $L^2$ (given it is a closed subspace), and so every orthnormal sequence in $M$ is countable. \\ \\
            Consider then an ONB for $M$ given by $\{f_k\}_{k = 1}^\infty$ (all Hilbert spaces admit orthonormal bases). We get the following:
            \begin{align}
                \sum_{k = 1}^\infty \langle f_k, f_k \rangle = \sum_{k = 1}^\infty ||f_k||_2^2 = \sum_{k = 1}^\infty \int_0^1 |f_k|^2 \\
                = \int_0^1 \sum_{k = 1}^\infty |f_k|^2 \qquad \text{(Swapping sum and integral, as } |f_k| > 0) \\
                \leq \int_0^1 C^2 = C^2 \qquad \text{(Using previous work)}
            \end{align}
            But of course $\sum_{k = 1}^\infty \langle f_k, f_k \rangle$ should be the amount of nonzero terms in $\{f_k\}_{k = 1}^\infty$, as its orthonormal. \\ \\Therefore the amount of nonzero terms is bounded above by $C^2$, i.e. we have a basis for $M$ with at most $C^2$ vectors. Thus the dimension of $M$ is at most $C^2$.
        \end{proof}
    \end{enumerate}
\end{ex}

\begin{ex}{20}
    Suppose $||f_0||_p = ||f_1||_p = 1$ and let:
    \begin{equation}
        f_t = (1 - t)f_0 + f_1
    \end{equation}
    be the straight line segment joining the points $f_0$ and $f_1$. Then $||f_t||_p < 1$ for all $t$ with $0 < t < 1$, unless $f_0 = f_1$.
    \begin{enumerate}[label=(\alph*)]
        \item Let $f \in L^p$ and $g \in L^q$, $p, q$ dual, with $||f||_p = 1$ and $||g||_q = 1$. Then:
        \begin{equation}
            \int fg d\mu = 1
        \end{equation}
        only when $f(x) = \text{sign}|g(x)|^{q-1}$.
        \begin{proof}
            We first verify that taking $f(x) = \text{sign}g(x)|g(x)|^{q-1}$ has that $\int fg d\mu = 1$, which we can do simply by noting $|g| = \text{sign}g \cdot g$.
            \begin{equation}
                \begin{aligned}
                    \int fg d\mu = \int \Big (\text{sign}g |g|^{q-1} \Big ) g d\mu \\
                    = \int \Big (\text{sign}g \cdot g)|g|^{q-1} d\mu = \int |g||g|^{q-1} d\mu \\
                    = \int |g|^q d\mu = ||g||_q^q = 1
                \end{aligned}
            \end{equation}
            Assume then for the other direction that $\int fg d\mu = 1$, we will show that $f(x) = \text{sign}|g(x)|^{q-1}$, where equality here is equality almost everywhere. \\ \\
            By applying H\"older's inequalty as well as monotonicity, we get the following:
            \begin{equation}
                1 = \int fg d\mu \leq \int |fg| d\mu = ||fg||_1 \leq ||f||_p||g||_q = 1
            \end{equation}
            Thus $||fg|| = ||f||_p||g||_q$. Using \textbf{Theorem 6.2} from Folland then, we know that equality holds in H\"older's inequality if and only $|f|^p = C|g|^q$ almost everywhere for some constant $C > 0$. \\ \\
            Thus we take $|f|^p = C|g|^q$ almost everywhere for some $C > 0$. Note that as $p, q$ dual we have $\frac{1}{p} + \frac{1}{q} = 1 \rightarrow \frac{q}{p} = q - 1$. We can therefore reorganize to get the following:
            \begin{equation}
                |f| = (C|g|^q)^{\frac{1}{p}} = C^{\frac{1}{p}} |g|^{\frac{q}{p}} = C^{\frac{1}{p}} |g|^{q-1}
            \end{equation}
            Recall then that we showed in (48) that $\int fg d\mu =  \int |fg| d\mu$. It is not particularly difficult to show that this implies $fg = |fg|$ a.e., which can seen in the following:
            \begin{equation}
                \begin{aligned}
                    \int fg d\mu =  \int |fg| d\mu \\
                    \rightarrow \int (fg)^+ - (fg)^- d\mu = \int (fg)^+ + (fg)^- \\
                    \rightarrow \int (fg)^+ d\mu - \int (fg)^- d\mu = \int (fg)^+ d\mu + \int (fg)^- d\mu \\
                    \rightarrow \int (fg)^- = 0
                \end{aligned}
            \end{equation}
            Which clearly implies $(fg)^- = 0$ a.e., i.e. $fg$ is at most negative on a set of measure zero, so clearly $fg = |fg|$ almost everywhere. With this in mind, note we additionally have following:
            \begin{equation}
                1 = ||f||_p^p = \int |f|^p = \int C |g|^q = C\int |g|^q = C||g||_q^q = C
            \end{equation}
            So $C = 1$, i.e. $|f| = |g|^{q-1}$ almost everywhere. With this, we can finally put everything together, also using the fact that $g = \text{sign}g|g|$:
                \begin{align}
                |f| = |g|^{q-1} \text{ a.e.}\\
                \rightarrow |fg| = |g|^q \text{ a.e.} \qquad \text{(Multiplying both sides by } |g|^q) \\
                \rightarrow fg = |g|^q \text{ a.e.} \qquad \text{(Using that } fg = |fg| \text{ a.e.}) \\
                \rightarrow f = \frac{|g|^q}{g} \text{ a.e.} \qquad \text{(Dividing by } g ) \\
                \end{align}
                Note then that as $\frac{1}{\text{sign} g} = \text{sign} g$, we have that:
                \begin{equation}
                    \frac{1}{g} = \frac{1}{\text{sign}g|g|} = \frac{\text{sign}g}{|g|}
                \end{equation}
                Using this with (55) finally gets:
                \begin{equation}
                    f = \frac{|g|^q}{g} = \frac{|g|^q \text{sign} g}{|g|} = \text{sign} g |g|^{q-1} \text{ a.e.}
                \end{equation}
                Note here some of our work relies on $g \neq 0$ so that division is defined, but we note when $g(x) = 0$ we have $f(x) = 0$, which is still consistent with $f(x) = \text{sign}g(x)|g(x)|^{q-1}$.
        \end{proof}
        \item Suppose that $||f_{t'}||_p = 1$ for some $0 < t' < 1$. Find some $g \in L^q$, $||g||_q = 1$, so that:
        \begin{equation}
            \int f_{t'}g d\mu = 1
        \end{equation}
        and let $F(t) = \int f_tg d\mu$. Observe as a result that $F(t) = 1$ for all $0 \leq t \le1 $. Conclude that $f_t = f_0$ for all $0 \leq t \leq 1$.
        \begin{proof}
            Let $g = \text{sign} f_{t'}|f_{t'}|^{p-1}$. We note here that this has $|g| = |f_{t'}|^{p-1}$, where we get the following:
            \begin{equation}
                \begin{aligned}
                    \int |g|^q d\mu = \int |f_{t'}|^{qp - q}
                \end{aligned}
            \end{equation}
            As $p, q$ dual, we note that we have $q + p = qp$. Thus $qp - q = p$, and so we continue further:
            \begin{equation}
                \begin{aligned}
                    \int |g|^q d\mu = \int |f_{t'}|^{qp - q} d\mu = \int |f_{t'}|^p  = ||f_{t'}||_p^p = 1
                \end{aligned}
            \end{equation}
            Thus $||g||^q_q = 1$, and so $||g||_q = 1$. Of course this also has $g \in L^q$, so this is the desired $g$. \\ \\
            We turn now to the given function $F(t)$, which we can examine in the following way:
            \begin{equation}
                \begin{aligned}
                    F(t) = \int f_t g d\mu = \int (1-t)f_0g + f_1g d\mu \\
                    = (1-t)\int f_0 g d\mu + t\int f_1 g d\mu
                \end{aligned}
            \end{equation}
            But $\int f_0 g d\mu$ and $\int f_1 g d\mu$ are constants, so it follows $F(t)$ is a linear polynomial in $t$. Moreover, we observe through an application of H\"older's inequality (and the triangle inequality) that we have the following:
            \begin{equation}
                \begin{aligned}
                    F(t) =\int f_t g d\mu \leq \int |f_t g| d\mu = ||f_t g||_1 \leq ||f_t||_p||g||_q \\
                    = ||f_t||_p = ||(1 - t)f_0 + tf_1||_p \leq (1-t)||f_0||_p + ||f_1||_p = 1 - t + t = 1
                \end{aligned}
            \end{equation}
            Putting it all together, we have that $F(t)$ is a linear polynomial in $t$, $F(t) \leq 1$ on $[0,1]$, and $F(t') = 1$ for $t' \in (0,1)$. \\ \\
            As $F(t)$ is a linear polynomial, it is either strictly increasing/decreasing or constant, as observed by taking its derivative. However, we note either of the strictly increasing/decreasing cases are not possible.
            \\ \\ In particular, this would have that for some small $\epsilon > 0$, $F(t' - \epsilon)$ or $F(t' + \epsilon)$ would be greater than $F(t')$ (depending on decreasing or increasing), but this would contradict that $F(t) \leq 1 = F(t')$ on $[0,1]$.  \\ \\
            Thus the only possibility is that $F(t)$ is a constant function equal to $1$. \\ \\
            We conclude that $f_t = f_0$ for all $0 \leq t \leq 1$ using part (a) then; in particular, note that having $F(t) = 1$ gives us that $||f_t||_p = 1$ for all $t \in [0,1]$ using our argument in (68). \\ \\
            Thus all the hypotheses of (a) are satisfied, so it is that $f_t = \text{sign}g|g|^{q-1} = f_0$, applying (a) to an arbitrary $t$ and $0$. This gives the desired conclusion.
        \end{proof}
        \item Show that the strict convexity fails when $p = 1$ or $p = \infty$. What can be said about these cases?
        \begin{proof}
            We present two counterexamples for the cases where $p = 1, p = \infty$. In the case of $p = 1$, we consider on $L^1([-1,1])$ the following functions: $$f_0 = \frac{1}{2}, f_1 = |x|$$
            Clearly we have $||f_0||_1 = ||f_1|| = 1$. However, we consider the following for some $0 < t < 1$:
            \begin{equation}
                \begin{aligned}
                    ||f_t||_1 = \int_{-1}^1 |f_t| d\mu = \int_{-1}^1 f_t d\mu \\
                    = \int_{-1}^1 (1-t)\frac{1}{2} + t|x| d\mu = (1-t)\int_{-1}^1 \frac{1}{2} + t \int_{-1}^1 |x| \\
                    (1 - t) + t = 1
                \end{aligned}
            \end{equation}
            And so strict convexity is violated. \\ \\
            For the case $p = \infty$, we work on $L^\infty([-1,1])$. We consider the following functions:
            \begin{equation}
                f_0 = \mathbbm{1}_{[-1,0]}, f_1 = 1
            \end{equation}
            We clearly have $||f_0||_\infty = ||f_1||_\infty = 1$. However, considering some point $x \in [-1,0]$, we note for some $0 < t < 1$:
            \begin{equation}
                f_t(x) = (1-t)\mathbbm{1}_{[-1,0]}(x) + t = (1 - t) + t = 1
            \end{equation}
            We know $[-1,0]$ is positive measure then, so  it follows $||f_t||_\infty \geq 1$, but this also violates strict convexity. \\ \\
            As for these cases, all we can say is that we have convexity, which follows from Minkowski's inequality.
        \end{proof}
    \end{enumerate}
\end{ex}



\end{document}